\documentclass[12pt]{article}

\usepackage{cmap}
\usepackage[T2A]{fontenc}
\usepackage[utf8]{inputenc}
\usepackage[russian]{babel}
\usepackage{graphicx}
\usepackage{amsthm,amsmath,amssymb}
\usepackage[russian,colorlinks=true,urlcolor=red,linkcolor=blue]{hyperref}
\usepackage{enumerate}
\usepackage{datetime}
\usepackage{minted}
\usepackage{fancyhdr}
\usepackage{lastpage}
\usepackage{color}
\usepackage{verbatim}
\usepackage{tikz}
\usepackage{epstopdf}

\def\NAME{Динамика}
\def\DATE{15 ноября}
\def\CURNO{\NO\t{9}}

\parskip=0em
\parindent=0em

\sloppy
\voffset=-20mm
\textheight=235mm
\hoffset=-25mm
\textwidth=180mm
\headsep=12pt
\footskip=20pt

\setcounter{page}{0}
\pagestyle{empty}

% Основные математические символы
\DeclareSymbolFont{extraup}{U}{zavm}{m}{n}
\DeclareMathSymbol{\heart}{\mathalpha}{extraup}{86}
\newcommand{\N}{\mathbb{N}}   % Natural numbers
\newcommand{\R}{\mathbb{R}}   % Ratio numbers
\newcommand{\Z}{\mathbb{Z}}   % Integer numbers
\def\EPS{\varepsilon}         %
\def\SO{\Rightarrow}          % =>
\def\EQ{\Leftrightarrow}      % <=>
\def\t{\texttt}               % mono font
\def\c#1{{\rm\sc{#1}}}        % font for classes NP, SAT, etc
\def\O{\mathcal{O}}           %
\def\NO{\t{\#}}               % #
\renewcommand{\le}{\leqslant} % <=, beauty
\renewcommand{\ge}{\geqslant} % >=, beauty
\def\XOR{\text{ {\raisebox{-2pt}{\ensuremath{\Hat{}}}} }}
\newcommand{\q}[1]{\langle #1 \rangle}               % <x>
\newcommand\URL[1]{{\footnotesize{\url{#1}}}}        %
\newcommand{\sfrac}[2]{{\scriptstyle\frac{#1}{#2}}}  % Очень маленькая дробь
\newcommand{\mfrac}[2]{{\textstyle\frac{#1}{#2}}}    % Небольшая дробь
\newcommand{\score}[1]{{\bf\color{red}{(#1)}}}

% Отступы
\def\makeparindent{\hspace*{\parindent}}
\def\up{\vspace*{-0.3em}}
\def\down{\vspace*{0.3em}}
\def\LINE{\vspace*{-1em}\noindent \underline{\hbox to 1\textwidth{{ } \hfil{ } \hfil{ } }}}
%\def\up{\vspace*{-\baselineskip}}

\lhead{Математический анализ}
\chead{}
\rhead{}
\renewcommand{\headrulewidth}{0.4pt}

\lfoot{}
\cfoot{\thepage\t{/}\pageref*{LastPage}}
\rfoot{}
\renewcommand{\footrulewidth}{0.4pt}

\newenvironment{MyList}[1][4pt]{
  \begin{enumerate}[1.]
  \setlength{\parskip}{0pt}
  \setlength{\itemsep}{#1}
}{       
  \end{enumerate}
}
\newenvironment{InnerMyList}[1][0pt]{
  \vspace*{-0.5em}
  \begin{enumerate}[a)]
  \setlength{\parskip}{#1}
  \setlength{\itemsep}{0pt}
}{
  \end{enumerate}
}

\newcommand{\Section}[1]{
  \refstepcounter{section}
  \addcontentsline{toc}{section}{\arabic{section}. #1} 
  %{\LARGE \bf \arabic{section}. #1} 
  {\LARGE \bf #1} 
  \vspace*{1em}
  \makeparindent\unskip
}
\newcommand{\Subsection}[1]{
  \refstepcounter{subsection}
  \addcontentsline{toc}{subsection}{\arabic{section}.\arabic{subsection}. #1} 
  {\Large \bf \arabic{section}.\arabic{subsection}. #1} 
  \vspace*{0.5em}
  \makeparindent\unskip
}
% Код с правильными отступами
\newenvironment{code}{
  \VerbatimEnvironment

  \vspace*{-0.5em}
  \begin{minted}{c}}{
  \end{minted}
  \vspace*{-0.5em}

}

% Формулы с правильными отступами
\newenvironment{smallformula}{
 
  \vspace*{-0.8em}
}{
  \vspace*{-1.2em}
  
}
\newenvironment{formula}{
 
  \vspace*{-0.0em}
}{
  \vspace*{-0.0em}
  
}

\definecolor{dkgreen}{rgb}{0,0.6,0}
\definecolor{brown}{rgb}{0.5,0.5,0}
\newcommand{\red}[1]{{\color{red}{#1}}}
\newcommand{\dkgreen}[1]{{\color{dkgreen}{#1}}}

\begin{document}
\newcommand\tab[1][0.5cm]{\hspace*{#1}}
\renewcommand{\dateseparator}{--}

\begin{center}
  {\Large\bf 
   Тригонометрия}\\
\end{center}

\vspace{-1em}
\LINE
\vspace{1em}
\pagestyle{fancy}
\begin{formula}
$
\sin^2 x + \cos^2 x = 1 \hspace{5ex}
\tg x \cdot \ctg x = 1 \hspace{5ex}
1 + \tg^2 x = \dfrac{1}{\cos^2 x}\hspace{5ex}
1 + \ctg^2 x = \dfrac{1}{\sin^2 x}\\
\\
\sin 2x = 2 \sin x \cos x \hspace{5ex} 
\cos 2x = \cos^2 x - \sin^2 x \hspace{5ex} 
\tg 2x = \dfrac{2\tg x}{1 - \tg^2 x} \hspace{5ex} 
\ctg 2x = \dfrac{\ctg^2 x - 1}{2 \ctg x}\\
\\
\sin 3x = 3\sin x - 4\sin^3 x \hspace{5ex} \cos 3x = 4\cos^3 x - 3\cos x\\
\\
\tg 3x = \dfrac{3\tg x - \tg^3 x}{1 - 3 \tg^2 x} \hspace{5ex} 
\ctg 3x = \dfrac{3\ctg x - \ctg^3 x}{1 - 3 \ctg^2 x}\\
\\
\sin(x + y) = \sin x \cos y + \cos x \sin y \hspace{5ex} 
\sin(x - y) = \sin x \cos y - \cos x \sin y \\
\\
\cos(x + y) = \cos x \cos y - \sin x \sin y \hspace{5ex} 
\cos(x - y) = \cos x \cos y + \sin x \sin y \\
\\
\tg(x + y) = \dfrac{\tg x + \tg y}{1 - \tg x \tg y} \hspace{5ex} 
\tg(x - y) = \dfrac{\tg x - \tg y}{1 + \tg x \tg y} \\
\\
\ctg(x + y) = \dfrac{\ctg x \ctg y - 1}{\ctg x + \ctg y} \hspace{5ex} 
\ctg(x - y) = \dfrac{\ctg x \ctg y + 1}{\ctg x - \ctg y} \\
\\
\sin x + \sin y = 2 \sin \dfrac{x + y}{2} \cos \dfrac{x - y}{2} \hspace{5ex} 
\sin x - \sin y = 2 \sin \dfrac{x - y}{2} \cos \dfrac{x + y}{2} \\
\\ 
\cos x + \cos y = 2 \cos \dfrac{x + y}{2} \cos \dfrac{x - y}{2} \hspace{5ex} 
\cos x + \cos y = -2 \sin \dfrac{x + y}{2} \sin \dfrac{x - y}{2} \\
\\
\sin x \sin y = \dfrac{1}{2}[\cos(x - y) - \cos(x + y)]\\
\\
\sin x \cos y = \dfrac{1}{2}[\sin(x + y) + \sin(x - y)]\\
\\
\cos x \cos y = \dfrac{1}{2}[\cos(x + y) + \cos(x - y)]\\
\\
\sin^2 x = \dfrac{1 - \cos 2x}{2} \hspace{5ex}
\cos^2 x = \dfrac{1 + \cos 2x}{2}\\
\\
\sin x = \dfrac{2\tg \frac{x}{2}}{1 + \tg^2 \frac{x}{2}} \hspace{5ex}
\cos x = \dfrac{1 - \tg^2 \frac{x}{2}}{1 + \tg^2 \frac{x}{2}} \hspace{5ex}
\tg x = \dfrac{2\tg \frac{x}{2}}{1 - \tg^2 \frac{x}{2}} \hspace{5ex}
\tg x = \dfrac{1 - \tg^2 \frac{x}{2}}{2\tg \frac{x}{2}} \\
\\
$
\end{formula}

\newpage

\begin{center}
  {\Large\bf 
   Геометрия}\\
\end{center}

\vspace{-1em}
\LINE
\vspace{1em}
\pagestyle{fancy}
\begin{formula}
$
S = \sqrt{p(p - a)(p - b)(p - c)}, p = \dfrac{a + b + c}{2}\\
\\
S = \dfrac{1}{3}\sqrt{M(M - m_a)(M - m_b)(M - m_c)}, M = \dfrac{m_a + m_b + m_c}{2}\\
\\
S = \dfrac{1}{\sqrt{H(H - \dfrac{2}{h_a})(H - \dfrac{2}{h_b})(H - \dfrac{2}{h_c})}}, H = \dfrac{1}{h_a} + \dfrac{1}{h_b} + \dfrac{1}{h_c}\\
\\
\lambda = \dfrac{m}{n}(\dfrac{p}{q} + 1)\\
\\
S = \dfrac{abc}{4R} = \dfrac{a^2 \sin B \sin C}{\sin A} = 2R^2\sin A \sin B \sin C\\
\\
m_a = \dfrac{1}{2}\sqrt{2c^2 + 2b^2 - a^2}\\
\\
l_a = \dfrac{bc\sin A}{(b + c) \sin \dfrac{A}{2}} = \sqrt{bc - a_b a_c}\\
\\
r = \dfrac{S}{p - d} $(вневписанная)$\\
\\
a^2 = b^2 + c^2 - 2bc \cos A\\
\\
\dfrac{a}{\sin A} = \dfrac{b}{\sin B} = \dfrac{c}{\sin C} = 2R\\
\\
\cos \alpha = \dfrac{x_1 x_2 + y_1 y_2}{\sqrt{x_1^2 + y_1^2}\sqrt{x_2^2 + y_2^2}}\\
\\ 
$
\end{formula}

\newpage

\begin{center}
  {\Large\bf 
   Неравенства}\\
\end{center}

\vspace{-1em}
\LINE
\vspace{1em}
\pagestyle{fancy}
\begin{formula}
Неравенсво Коши: $x_1, x_2, \dots x_n \in \R^+$\\

$\dfrac{n}{\dfrac{1}{x_1} + \dfrac{1}{x_2} + \dots + \dfrac{1}{x_n}} \le (x_1 \cdot x_2 \cdot \hdots \cdot x_n )^{1/n} \le \dfrac{x_1 + x_2 + \dots + x_n}{n} \le \sqrt{\dfrac{x_1^2 + x_2^2 + \dots + x_n^2}{n}}$\\

Неравенство Бернулли: $x > -1, n \in \N$; $(1 + x)^n \ge 1 + nx$

\end{formula}

\begin{center}
  {\Large\bf 
   Замечательные пределы}\\
\end{center}

\vspace{-1em}
\LINE
\vspace{1em}
\pagestyle{fancy}
\begin{formula}
	$
	\lim \limits_{x \rightarrow 0} \dfrac{\sin x}{x} = 1 \hspace{10ex} 
	\lim \limits_{x \rightarrow 0} \dfrac{\tg x}{x} = 1 \hspace{10ex} 
	\lim \limits_{x \rightarrow 0} \dfrac{\sh x}{x} = 1 \hspace{10ex} 
	\lim \limits_{x \rightarrow 0} \dfrac{\th x}{x} = 1 \\
	\\
	\lim \limits_{x \rightarrow 0} \dfrac{1 - \cos x}{x^2} = \dfrac{1}{2} \\
	\\
	\lim \limits_{x \rightarrow 0} \dfrac{\arcsin x}{x} = 1 \hspace{7ex} 
	\lim \limits_{x \rightarrow 0} \dfrac{\arctg x}{x} = 1 \\
	\\
	\lim \limits_{x \rightarrow 0} (1 + x)^{1/x} = e \hspace{5ex} 
	\lim \limits_{x \rightarrow +\infty} (1 + \frac{1}{x})^{x} = e \\
	\\
	\lim \limits_{x \rightarrow 0} \dfrac{\ln x}{x} = 1 \hspace{11ex} 
	\lim \limits_{x \rightarrow 0} \dfrac{a^x - 1}{x} = \ln a \\
	\\
	\lim \limits_{x \rightarrow 0} \dfrac{(1 + x)^p - 1}{x} = p
	$
\end{formula}

\begin{center}
  {\Large\bf 
   Гиперболические функции}\\
\end{center}

\vspace{-1em}
\LINE
\vspace{1em}
\pagestyle{fancy}
\begin{formula}
$\sh x = \dfrac{e^x - e^{-x}}{2} \hspace{10ex} \ch x = \dfrac{e^x + e^{-x}}{2} \hspace{10ex}$ 
$\ch ^2 x - sh^2 x = 1$\\
\end{formula}
\newpage
\begin{center}
  {\Large\bf 
   Ряды Тейлора}\\
\end{center}

\vspace{-1em}
\LINE
\vspace{1em}
\pagestyle{fancy}
\begin{formula}
$e^x = 1 + x + \dfrac{x^2}{2!} + \dfrac{x^3}{3!} + \dfrac{x^4}{4!} + ... + \dfrac{x^n}{n!} + o(x^n)$\\
\\
$\sh x = x + \dfrac{x^3}{3!} + \dfrac{x^5}{5!} + \dfrac{x^7}{7!} + ... + \dfrac{x^{2n + 1}}{(2n + 1)!} + o(x^{2n + 1})$\\
\\
$\ch x = 1 + \dfrac{x^2}{2!} + \dfrac{x^4}{4!} + \dfrac{x^6}{6!} + ... + \dfrac{x^{2n}}{(2n)!} + o(x^{2n})$\\
\\
$\sin x = x - \dfrac{x^3}{3!} + \dfrac{x^5}{5!} - \dfrac{x^7}{7!} + ... + (-1)^n\dfrac{x^{2n + 1}}{(2n + 1)!} + o(x^{2n + 1})$\\
\\
$\cos x = 1 - \dfrac{x^2}{2!} + \dfrac{x^4}{4!} - \dfrac{x^6}{6!} + ... + (-1)^n\dfrac{x^{2n}}{(2n)!} + o(x^{2n})$\\
\\
$\tg x = x + \dfrac{1}{3}x^3 + \dfrac{2}{15}x^5 + o(x^5)$\\
\\
$\ln(1 + x) = x - \dfrac{x^2}{2} + \dfrac{x^3}{3} - \dfrac{x^4}{4} + \dots + (-1)^{n + 1}\dfrac{x^n}{n} + o(x^n)$\\
\\
$(1 + x)^p = 1 + px + \dfrac{p(p - 1)}{2!}x^2 + \dfrac{p(p - 1)(p - 2)}{3!}x^3 + \dots + \dfrac{p(p - 1)\dots(p - n + 1)}{n!}x^n + o(x^n)$\\
\\
$\arctg x = x - \dfrac{x^3}{3} + \dfrac{x^5}{5} - \dfrac{x^7}{7} + ... + (-1)^n\dfrac{x^{2n + 1}}{2n + 1} + o(x^{2n + 1})$\\
\\
$\arcsin x = x + \dfrac{x^3}{6} + \dfrac{3x^5}{40} + \dfrac{5x^7}{112} + o(x^{7})$\\
\\

\end{formula}

\begin{center}
  {\Large\bf 
   Производные}\\
\end{center}

\vspace{-1em}
\LINE
\vspace{1em}
\pagestyle{fancy}
\begin{formula}
$
	(x^n)' = nx^{n - 1} \hspace{5ex} 
	(\sin x)' = \cos x \hspace{5ex} 
	(\cos x)' = -\sin x \hspace{5ex} 
	(\tg x)' = \dfrac{1}{\cos^2 x} \hspace{5ex} \\
	(\ctg x)' = -\dfrac{1}{\sin^2 x} \hspace{5ex}
	(e^x)' = e^x \hspace{5ex} 
	(a^x)' = a^x \ln a \hspace{5ex} 
	(\ln |x|)' = \dfrac{1}{x} \hspace{5ex} 
	(\log_a |x|)' = \dfrac{1} {x \ln a} \hspace{5ex}\\
	(\arcsin x)' = \dfrac{1}{\sqrt{1 - x^2}} \hspace{5ex}
	(\arccos x)' = -\dfrac{1}{\sqrt{1 - x^2}} \hspace{5ex}
	(\arctg x)' = \dfrac{1}{1 + x^2} \hspace{5ex}\\
	(\arcctg x)' = -\dfrac{1}{1 + x^2} \hspace{5ex}
	(\sh x)' = \ch x \hspace{5ex} 
	(\ch x)' = \sh x \hspace{5ex} 
	(\th x)' = \dfrac{1}{\ch^2 x} \hspace{5ex} \\
	(\cth x)' = -\dfrac{1}{\sh^2 x} \hspace{5ex}	\\
$
\end{formula}

\begin{formula}
$
	(af(x) + bg(x))' = af'(x) + bg'(x) \hspace{5ex}
	(f(x)g(x))' = f'(x)g(x) + f(x)g'(x)\\
	\left ( \dfrac{f(x)}{g(x)} \right ) ' = \dfrac{f'(x)g(x) - f(x)g'(x)}{g^2(x)} \hspace{5ex}
	(f(g(x)))' = f'(g(x))g'(x)\\
	(u(x)v(x))^{(n)} = \sum \limits_{k = 0}^n {n \choose k}u^{(k)}(x)v^{(n - k)}(x )
$
\end{formula}

\newpage
\begin{center}
  {\Large\bf 
   Интегралы}\\
\end{center}

\vspace{-1em}
\LINE
\vspace{1em}
\pagestyle{fancy}
\begin{formula}
$
\int x^n dx = \dfrac{x^{n + 1}}{n + 1} + C \hspace{5ex} \\
\int \dfrac{dx}{x} = \ln |x| + C \\
\\
\int \dfrac{dx}{x^2 + a^2} = \dfrac{1}{a} \arctg \dfrac{x}{a} + C \hspace{5ex}\\
\int \dfrac{dx}{x^2 - a^2} = \dfrac{1}{2a} \ln \left | \dfrac{x - a}{x + a} \right |+ C \hspace{5ex}\\
\int \dfrac{dx}{\sqrt{a^2 - x^2}} = \arcsin \dfrac{x}{a} + C\\
\int \dfrac{dx}{\sqrt{x^2 - a^2}} = \ln |x + \sqrt{x^2 - a^2}| + C\\
\int \dfrac{dx}{\sqrt{x^2 + a}} = \ln |x + \sqrt{x^2 + a}| + C\\
\\
\int \sin x dx = -\cos x + C \hspace{5ex} \\
\int \cos x dx = \sin x + C \hspace{5ex} \\
\int \dfrac{dx}{\sin^2 x} = -ctg x + C \\
\int \dfrac{dx}{\cos^2 x} = tg x + C \hspace{5ex} \\
\int \dfrac{dx}{\sin x} = \ln |\tg \dfrac{x}{2}| + C \hspace{5ex} \\
\int \dfrac{dx}{\cos x} = -\ln |\tg (\dfrac{x}{2} - \dfrac{\pi}{4})| + C \\
 \int \tg x dx = -\ln |\cos x| + C \hspace{5ex} \\
 \int \ctg x dx = \ln |\sin x| + C\\
 \\
 \int e^x dx = e^x + C \hspace{5ex} \\
 \int a^x dx = \dfrac{a^x}{\ln a} + C \hspace{5ex} \\
 \int \sh x dx = \ch x + C \hspace{5ex}\\
\int \ch x dx = \sh x + C \\
\int \ln x dx = x\ln x - x + C \\
\\
\int \sqrt{x^2 - a^2}dx = \frac{x}{2}\sqrt{x^2 - a^2} + \frac{a^2}{4}\ln|\frac{\sqrt{x^2 - a^2} - x}{\sqrt{x^2 - a^2} + x}| + C\\
\int \sqrt{a^2 - x^2}dx = \frac{x}{2}\sqrt{a^2 - x^2} + \frac{a^2}{2}\arctg{\frac{x}{\sqrt{a^2 - x^2}}} + C\\
\int \sqrt{x^2 + a^2}dx = \frac{a^2}{2}\ln{|x + \sqrt{x^2 + a^2}|} + \frac{x}{2}\sqrt{x^2 + a^2} + C\\
\\
%d(f(x)) = f'(x)dx\\
%\int f(g(x))g'(x)dx = F(g(x)) + C = \int f(g(x))d(g(x)) = \int f(t)dt = F(t) + C = F(g(t)) + C\\
%\int f(x)g'(x)dx = f(x)g(x) - \int g(x)f'(x)dx = \int f(x)d(g(x)) = f(x)g(x) - \int g(x)d(f(x))\\
 $
\end{formula}
\end{document}
